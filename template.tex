\documentclass[letterpaper,11pt]{article}

\usepackage{latexsym}
\usepackage[empty]{fullpage}
\usepackage{titlesec}
\usepackage{marvosym}
\usepackage[usenames,dvipsnames]{color}
\usepackage{verbatim}
\usepackage{enumitem}
\usepackage[colorlinks=true, linkcolor=blue, citecolor=blue, urlcolor=blue]{hyperref}
\usepackage[english]{babel}
\usepackage{tabularx}
\usepackage{fontawesome5}
\usepackage{multicol}
\usepackage{graphicx}
\setlength{\multicolsep}{-3.0pt}
\setlength{\columnsep}{-1pt}
\input{glyphtounicode}

\RequirePackage{tikz}
\RequirePackage{xcolor}
\usepackage{tikz}
\usetikzlibrary{svg.path}


\definecolor{cvblue}{HTML}{0E5484}
\definecolor{black}{HTML}{130810}
\definecolor{darkcolor}{HTML}{0F4539}
\definecolor{cvgreen}{HTML}{3BD80D}
\definecolor{taggreen}{HTML}{00E278}
\definecolor{SlateGrey}{HTML}{2E2E2E}
\definecolor{LightGrey}{HTML}{666666}
\colorlet{name}{black}
\colorlet{tagline}{darkcolor}
\colorlet{heading}{darkcolor}
\colorlet{headingrule}{cvblue}
\colorlet{accent}{darkcolor}
\colorlet{emphasis}{SlateGrey}
\colorlet{body}{LightGrey}



%----------FONT OPTIONS----------
% sans-serif
% \usepackage[sfdefault]{FiraSans}
% \usepackage[sfdefault]{roboto}
% \usepackage[sfdefault]{noto-sans}
% \usepackage[default]{sourcesanspro}

% serif
% \usepackage{CormorantGaramond}
% \usepackage{charter}


% \pagestyle{fancy}
% \fancyhf{}  % clear all header and footer fields
% \fancyfoot{}
% \renewcommand{\headrulewidth}{0pt}
% \renewcommand{\footrulewidth}{0pt}

% Adjust margins
\addtolength{\oddsidemargin}{-0.6in}
\addtolength{\evensidemargin}{-0.5in}
\addtolength{\textwidth}{1.19in}
\addtolength{\topmargin}{-.7in}
\addtolength{\textheight}{1.4in}

\urlstyle{same}

\raggedbottom
\raggedright
\setlength{\tabcolsep}{0in}

% Sections formatting
\titleformat{\section}{
  \vspace{-4pt}\scshape\raggedright\large\bfseries
}{}{0em}{}[\color{black}\titlerule \vspace{-5pt}]

% Ensure that generate pdf is machine readable/ATS parsable
\pdfgentounicode=1

%-------------------------
% Custom commands
\newcommand{\resumeItem}[1]{
  \item\small{
    {#1 \vspace{-2pt}}
  }
}

\newcommand{\classesList}[4]{
    \item\small{
        {#1 #2 #3 #4 \vspace{-2pt}}
  }
}

\newcommand{\resumeSubheading}[4]{
  \vspace{-2pt}\item
    \begin{tabular*}{1.0\textwidth}[t]{l@{\extracolsep{\fill}}r}
      \textbf{\large#1} & \textbf{\small #2} \\
      \textit{\large#3} & \textit{\small #4} \\
      
    \end{tabular*}\vspace{-7pt}
}

\newcommand{\resumeSubSubheading}[2]{
    \item
    \begin{tabular*}{0.97\textwidth}{l@{\extracolsep{\fill}}r}
      \textit{\small#1} & \textit{\small #2} \\
    \end{tabular*}\vspace{-7pt}
}


\newcommand{\resumeProjectHeading}[2]{
    \item
    \begin{tabular*}{1.001\textwidth}{l@{\extracolsep{\fill}}r}
      \small#1 & \textbf{\small #2}\\
    \end{tabular*}\vspace{-7pt}
}

\newcommand{\resumeSubItem}[1]{\resumeItem{#1}\vspace{-4pt}}

\renewcommand\labelitemi{$\vcenter{\hbox{\tiny$\bullet$}}$}
\renewcommand\labelitemii{$\vcenter{\hbox{\tiny$\bullet$}}$}

\newcommand{\resumeSubHeadingListStart}{\begin{itemize}[leftmargin=0.0in, label={}]}
\newcommand{\resumeSubHeadingListEnd}{\end{itemize}}
\newcommand{\resumeItemListStart}{\begin{itemize}}
\newcommand{\resumeItemListEnd}{\end{itemize}\vspace{-5pt}}


\newcommand\sbullet[1][.5]{\mathbin{\vcenter{\hbox{\scalebox{#1}{$\bullet$}}}}}



% Command for inline heading
\newcommand{\resumeSubheadingInline}[3]{
  \item \textbf{#1} -- \textit{#2} \hfill \textit{#3}
}
\newcommand{\resumeItemCompact}[1]{
  \item {\small #1}
}
\newcommand{\resumeItemListStartCompact}{\begin{itemize}[leftmargin=0.15in, topsep=0pt, partopsep=0pt, itemsep=1pt]}
\newcommand{\resumeItemListEndCompact}{\end{itemize}}




%-------------------------------------------
%%%%%%  RESUME STARTS HERE  %%%%%%%%%%%%%%%%%%%%%%%%%%%%


\begin{document}

%----------HEADER----------
\begin{center}
    {\Huge \scshape [Your Name]} \\[2pt]
    [City, State] \\[1pt]
    \small 
    \href{tel:[Your Phone]}{\faPhone\ [Your Phone]} ~
    \href{mailto:[Your Email]}{\faEnvelope\ [Your Email]} ~
    \href{https://linkedin.com/in/[Your-Profile]}{\faLinkedin\ [Your-Profile]} ~
    \href{https://github.com/[Your-Username]}{\faGithub\ [Your-Username]}
\end{center}
\vspace{-8pt}

%-----------SUMMARY / OBJECTIVE----------
\section*{Summary}
\small
[A brief, 2-3 sentence summary of your professional background, key skills, and career objective. Tailor this to the specific job you are applying for. For example: "Highly motivated Software Engineer with 5+ years of experience in building scalable web applications. Seeking to leverage expertise in Python and cloud computing to contribute to innovative projects."]

%-----------EXPERIENCE----------
\section{Experience}
\resumeSubHeadingListStart

\resumeSubheadingInline{[Company Name]}{[Your Job Title]}{[Start Date] – [End Date]}
\resumeItemListStartCompact
    \resumeItemCompact{[Describe your key accomplishment or responsibility. Use action verbs and quantify your results whenever possible. For example: "Developed a new feature that increased user engagement by 15%."]}
    \resumeItemCompact{[Add another bullet point for a different accomplishment or responsibility.]}
    \resumeItemCompact{[Keep it concise and focused on impact.]}
\resumeItemListEndCompact

\resumeSubheadingInline{[Previous Company Name]}{[Your Job Title]}{[Start Date] – [End Date]}
\resumeItemListStartCompact
    \resumeItemCompact{[Describe your key accomplishment or responsibility.]}
    \resumeItemCompact{[Describe another key accomplishment or responsibility.]}
\resumeItemListEndCompact

\resumeSubHeadingListEnd

%-----------TECHNICAL SKILLS----------
\section*{Technical Skills}
\small
\textbf{Core Concepts:} [e.g., Object-Oriented Design, Algorithm Design, Data Structures, Problem Solving] \\
\textbf{Programming:} [e.g., Python, Java, C++, JavaScript, SQL, HTML/CSS] \\
\textbf{Frameworks/Libraries:} [e.g., ReactJS, NodeJS, Django, Flask, PyTorch, TensorFlow] \\
\textbf{Tools/Platforms:} [e.g., Linux, Git, GitHub, Docker, Kubernetes, AWS, Azure]


%-----------EDUCATION----------
\section*{Education}
\small
\begin{tabular*}{\textwidth}{@{\extracolsep{\fill}} l r}
\textbf{[Your Degree, Major]}, [Your University], [CGPA/Percentage] & [Start Year] – [End Year] \\
\textbf{[Your High School Degree]}, [Your High School Name] & [Start Year] – [End Year] \\
\end{tabular*}

%-----------PUBLICATIONS----------
\section*{Publications}
\small
% Use this section if you have published papers. If not, you can remove it.
\textbf{[Title of Your Publication]} \hfill [Date] \\
[Conference / Journal Name] \\
[Brief description of your publication, your contribution, and its impact. Add a DOI link if available.] \href{[DOI Link]}{DOI} \\[1ex]

\textbf{[Another Publication Title]} \hfill [Date] \\
[Conference / Journal Name] \\
[Brief description.] \href{[DOI Link]}{DOI} \\[1ex]


%-----------ACHIEVEMENTS / AWARDS----------
\section*{Achievements / Awards}
\small
\begin{itemize}[leftmargin=0.15in, label={}]
    \item \textbf{[Award Name], [Awarding Organization]} – [e.g., National Level]
    \item \textbf{[Hackathon or Competition Win], [Event Name]} – [e.g., First Place]
    \item \textbf{[Scholarship or Grant Name]} – [Awarding Institution]
\end{itemize}

%-----------EXTRACURRICULAR / LEADERSHIP----------
\section*{Extracurricular / Leadership}
\small
\begin{itemize}[leftmargin=0.15in, label={}]
    \item \textbf{[Your Role] – [Organization Name], [City]}: [Briefly describe your responsibilities and achievements. For example: "Organized a workshop for 50+ students on web development, receiving 95\% positive feedback."]}
    \item \textbf{[Your Role] – [Another Organization], [City]}: [Describe your responsibilities and achievements in this role.]
    \item \textbf{[Volunteer Work] – [Organization Name], [City]}: [Describe your contributions and the impact of your work.]
\end{itemize}

\end{document}